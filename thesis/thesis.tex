\documentclass[]{article}

%opening
\title{Remote Batteryless Internet-of-Things Sensor Testbed}
\author{B.T. Blokland}

\begin{document}

\maketitle

\section{Introduction}

The Internet-of-Things (IoT) is a promising vision which enables trillions of sensor devices to be connected. A common bottleneck for such devices is the energy supply. Batteries are large, expensive, heavy and wear out after several years.

A more sustainable solution than batteries, is energy harvesting where a device collects it's energy from the environment i.e. solar, radio frequency (RF), thermal or kinetic energy.

However, developing software for such devices comes with a challenge. Environmental energy can be scarce, causing frequent power failures. This contrasts with the standard assumption that programs run continuously throughout execution. The programmer has to take care of this intermittent behavior by i.e. storing data to non-volatile memory at certain intervals. The available amount of energy tends to be random, making it hard to predict how long a program can execute before the next power failure.

Setting up a test environment 

\end{document}
